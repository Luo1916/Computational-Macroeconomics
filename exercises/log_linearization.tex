\section{Log-Linearization vs first-order perturbation\label{ex:LogLinearization}}

\subsection*{Model variants}
Let us consider the small scale New Keynesian model from \textcite{An.Schorfheide_2007_BayesianAnalysisDSGE},
  which is also covered in the text book by \textcite{Herbst.Schorfheide_2016_BayesianEstimationDSGE}.
The model features Rotemberg adjustment costs and in the paper the authors present three different model representations
  (refer to the original paper for the meaning of the variables and parameters).
For all different model variants we will use the same parametrization as given in Table 3 of the paper:
\begin{align*}
\tau&=2, \quad
\kappa=0.33, \quad
\psi_1 = 1.50, \quad
\psi_2  = 0.125, \quad
\rho_R  = 0.75, \quad
\rho_g  = 0.95, \quad
\rho_z  = 0.90,\\
r^A  &= 1.00, \quad
\pi^A = 3.20, \quad
\gamma^Q = 0.55, \quad
\nu = 0.10, \quad
\bar{g} = 1/0.85.
\end{align*}

\paragraph{Model A:\ original nonlinear model}
\begin{align}
1 &= \beta \mathbb{E}_t\left[{\left(\frac{C_{t+1}/A_{t+1}}{C_t/A_t}\right)}^{-\tau} \frac{A_t}{A_{t+1}} \frac{R_t}{\pi_{t+1}}\right]
\\
1 &= \phi \left(\pi_t - \pi\right) \left[\left(1-\frac{1}{2\nu}\right)\pi_t + \frac{\pi}{2\nu}\right] \nonumber
\\&\qquad\qquad\qquad
- \phi \beta \mathbb{E}_t \left[{\left(\frac{C_{t+1}/A_{t+1}}{C_t/A_t}\right)}^{-\tau} \frac{Y_{t+1}/A_{t+1}}{Y_t/A_t} \left(\pi_{t+1} - \pi \right) \pi_{t+1}\right] + \frac{1}{\nu}\left[1-{\left({\frac{C_t}{A_t}}\right)}^{\tau}\right]
\\
Y_t &= C_t + G_t + \frac{\phi}{2} {\left({\pi_t - \pi}\right)}^2 Y_t
\\
G_t &= \frac{g_t-1}{g_t} Y_t
\\
R_t &= {R_t^{*}}^{1-\rho_R} R_{t-1}^{\rho_R} e^{\epsilon_{R,t}}
\\
R_t^* & = R {\left(\frac{\pi_t}{\bar{\pi}}\right)}^{\psi_1} {\left(\frac{Y_t}{Y_t^*}\right)}^{\psi_2}
\\
\ln(A_t) &= \ln(\gamma) + \ln(A_{t-1}) + \ln(z_t)
\\
\ln(z_t) &= \rho_z \ln(z_{t-1}) + \epsilon_{z,t}
\\
\ln(g_t) &= (1-\rho_g)\ln(\bar{g}) + \rho_g \ln(g_{t-1}) + \epsilon_{g,t}
\\
Y_t^* &= {(1-\nu)}^{\frac{1}{\tau}} A_t g_t
\end{align}
The productivity shock \(\varepsilon_{z,t}\), the government spending shock \(\varepsilon_{g,t}\)
  and the monetary policy shock \(\varepsilon_{R,t}\) are iid Gaussian:
\begin{align*}
\begin{pmatrix}
\varepsilon_{z,t}\\\varepsilon_{g,t}\\\varepsilon_{R,t}
\end{pmatrix}
\sim N\left(\begin{pmatrix} 0\\0\end{pmatrix}, \begin{pmatrix} \sigma_z^2 & 0 & 0\\0 & \sigma_{g}^2 &0 \\ 0 & 0 & \sigma_R^2\end{pmatrix}\right)
\end{align*}

Moreover, we have the following auxiliary parameter relationships:
\begin{align*}
\gamma = 1+\frac{\gamma^{Q}}{100}, \qquad
\beta = \frac{1}{1+R^{A}/400}, \qquad
\bar{\pi} = 1+\frac{\pi^{A}}{400}, \qquad
\phi=\tau\frac{1-\nu}{\nu\bar{\pi}^2\kappa}
\end{align*}

\paragraph{Model B:\ stationary nonlinear model}
Let's denote: \(c_t= C_t/A_t\), \(y_t= Y_t/A_t\) and \(y^*_t= Y^*_t/A_t\), then Model A can be equivalently represented as:
\begin{align}
1 &= \beta \mathbb{E}_t\left[{\left(\frac{c_{t+1}}{c_t}\right)}^{-\tau} \frac{1}{\gamma z_{t+1}} \frac{R_t}{\pi_{t+1}}\right] \label{eq:AS_B1}
\\
1 &= \phi \left(\pi_t - \pi\right) \left[\left(1-\frac{1}{2\nu}\right)\pi_t + \frac{\pi}{2\nu}\right]
- \phi \beta \mathbb{E}_t \left[{\left(\frac{c_{t+1}}{c_t}\right)}^{-\tau} \frac{y_{t+1}}{y_t} \left(\pi_{t+1} - \pi \right) \pi_{t+1}\right] + \frac{1}{\nu}\left[1-c_t^{\tau}\right]
\\
y_t &= c_t + \frac{g_t-1}{g_t} y_t + \frac{\phi}{2} {\left({\pi_t - \pi}\right)}^2 y_t
\\
R_t &= {R_t^{*}}^{1-\rho_R} R_{t-1}^{\rho_R} e^{\epsilon_{R,t}}
\\
R_t^* & = R {\left(\frac{\pi_t}{\bar{\pi}}\right)}^{\psi_1} {\left(\frac{y_t}{y_t^*}\right)}^{\psi_2}
\\
\ln(z_t) &= \rho_z \ln(z_{t-1}) + \epsilon_{z,t}
\\
\ln(g_t) &= (1-\rho_g)\ln(\bar{g}) + \rho_g \ln(g_{t-1}) + \epsilon_{g,t}
\\
y_t^* &= {(1-\nu)}^{\frac{1}{\tau}} g_t
\end{align}
with the following auxiliary parameters:
\begin{align*}
  \gamma = 1+\frac{\gamma^{Q}}{100}, \qquad
  \beta = \frac{1}{1+R^{A}/400}, \qquad
  \bar{\pi} = 1+\frac{\pi^{A}}{400}, \qquad
  \phi=\tau\frac{1-\nu}{\nu\bar{\pi}^2\kappa}
\end{align*}
The steady-state is given by:
\begin{align*}
z=1, \qquad\pi = \bar{\pi}, \qquad g=\bar{g}, \qquad R=\frac{\gamma}{\beta}\pi, \qquad R^* = R, \qquad c = {(1-\nu)}^{\frac{1}{\tau}}, \qquad y = gc , \qquad y^*=y
\end{align*}

\paragraph{Model C:\ Exponential transform}
Let's denote hat variables as log deviations from steady-state: \(\hat{x}_t = \ln\left(\frac{X_t}{X}\right)\), then Model B can be equivalently represented as:
\begin{align}
1 &= \mathbb{E}_t \left[e^{-\tau \hat{c}_{t+1} + \tau \hat{c}_{t} + \hat{R}_{t} - \hat{z}_{t+1} - \hat{\pi}_{t+1} }\right]\label{eq:AS_C1}
\\
0 &= \left(e^{\hat{\pi}_{t}}-1\right) \left[\left(1-\frac{1}{2\nu}\right)e^{\hat{\pi}_{t}} + \frac{1}{2\nu}\right] \nonumber
\\&\qquad\qquad\qquad
- \beta \mathbb{E}_t \left[\left(e^{\hat{\pi}_{t+1}}-1 \right) e^{-\tau \hat{c}_{t+1} + \tau \hat{c}_{t} + \hat{y}_{t+1} - \hat{y}_{t} + \hat{\pi}_{t+1}}\right] + \frac{1-\nu}{\nu\pi^2\phi}\left(1-e^{\tau\hat{c}_{t}}\right)
\\
e^{\hat{c}_{t}-\hat{y}_{t}} &= e^{-\hat{g}_{t}} - \frac{\phi \pi^2 g}{2} {\left(e^{\hat{\pi}_{t}}-1\right)}^2\label{eq:AS_C3}
\\
\hat{R}_{t} &= \rho_R \hat{R}_{t-1} + (1-\rho_R) \psi_1 \hat{\pi}_{t} + (1-\rho_R)\psi_2(\hat{y}_{t}-\hat{g}_{t}) + \epsilon_{R,t}
\\
\hat{z}_{t} &= \rho_z \hat{z}_{t-1} + \epsilon_{z,t}
\\
\hat{g}_{t} &= \rho_g \hat{g}_{t-1} + \epsilon_{g,t}
\end{align}

\paragraph{Model D:\ Log-linearization}
Taking the first-order Taylor approximation in logged variables (also known as hat variables) yields:
\begin{align}
\hat{y}_{t} &= \mathbb{E}_t \hat{y}_{t+1} + \hat{g}_{t} - E_t\hat{g}_{t+1} - \frac{1}{\tau} \left(\hat{R}_{t}- \mathbb{E}_t \hat{\pi}_{t+1} - \mathbb{E}_t \hat{z}_{t+1}\right) \label{eq:AS_D1}
\\
\hat{\pi}_{t} &= \beta E_t\hat{\pi}_{t+1} + \kappa \left(\hat{y}_{t} - \hat{g}_{t}\right)
\\
\hat{c}_{t} &= \hat{y}_{t} - \hat{g}_{t}
\\
\hat{R}_{t} &= \rho_R \hat{R}_{t-1} + (1-\rho_R) \psi_1 \hat{\pi}_{t} + (1-\rho_R)\psi_2(\hat{y}_{t}-\hat{g}_{t}) + \epsilon_{R,t}
\\
\hat{g}_{t} &= \rho_g \hat{g}_{t-1} + \epsilon_{g,t}
\\
\hat{z}_{t} &= \rho_z \hat{z}_{t-1} + \epsilon_{z,t}
\end{align}
where \(\beta = \frac{1}{1+R^{A}/400}\) and \(\kappa=\tau\frac{1-\nu}{\nu\bar{\pi}^2\phi}\).

\paragraph{Measurement Equations}
The measurement equations are given by:
\begin{align}
{YGR}_t &= \gamma^{(Q)} + 100(\hat{y}_t-\hat{y}_{t-1} + \hat{z}_t)
\\
{INFL}_t &= \pi^{(A)} + 400 \hat{\pi}_t
\\
{INT}_t &= \pi^{(A)} + r^{(A)} + 4 \gamma^{(Q)} 400 \hat{R}_t
\end{align}

\subsection*{Exercises}
\begin{enumerate}
\item What are the differences of the model compared to the New Keynesian model we studied so far?

\item Would model A and model B generate the same trajectories for the endogenous variables?
  
\item Would model B and model C generate the same trajectories for the endogenous variables?
  
\item Derive (on paper)
\begin{itemize}
    \item equation~\eqref{eq:AS_C1} from equation~\eqref{eq:AS_B1}
    \item equation~\eqref{eq:AS_D1} by combining the first-order Taylor expansion of equations~\eqref{eq:AS_C1} and~\eqref{eq:AS_C3}
\end{itemize}

\item Write a mod file for Model B (including the measurement equations) in Dynare
  and solve it using a first-order perturbation approximation of the policy function.
Make note of the impulse response functions and of the theoretical moments.

\item Write a mod file for Model C (including the measurement equations) in Dynare
  and solve it using a first-order perturbation approximation of the policy function.
Make note of the impulse response functions and of the theoretical moments.
Compare the theoretical moments and explain the differences or equivalences to Model B.
  
\item Write a mod file for Model D (including the measurement equations) in Dynare,
  and solve it using a first-order perturbation approximation of the policy function.
Make note of the impulse response functions and of the theoretical moments
Compare the theoretical moments and explain the differences or equivalences with Model C.
  
\item What are the advantages and disadvantages of log-linearizing model equations by hand
  or by using first-order perturbation on the nonlinear model equations?
  
\end{enumerate}

\paragraph{Readings}
\begin{itemize}
	\item \textcite{An.Schorfheide_2007_BayesianAnalysisDSGE}
\end{itemize}

\begin{solution}\textbf{Solution to \nameref{ex:LogLinearization}}
\ifDisplaySolutions%
\begin{enumerate}

\item Compared to our baseline New Keynesian model, this model
    \begin{itemize}
    \item has the Rotemberg-pricing assumption instead of Calvo price rigidities.
    \item a linear production function instead of a Cobb-Douglas one.
    \item no capital and no investment
    \item includes a stochastic process for fiscal policy.
    \item includes a persistence term in the Taylor rule.
    \item has a unit root in total factor productivity and a shock to the growth of TFP.
\end{itemize}
  
\item Note that $A_t$ has a unit root, so in model A the variables $Y_t$, $Y_t^*$, $C_t$ and $G_t$ would be trending,
  while in model B the de-trended variables $y_t$, $y_t^*$, $c_t$ and $g_t$ would be stationary.
It is important to work with the stationary model in Dynare to have a well-behaved (constant!) steady-state
  around which the perturbation solution is computed.

\item No, variables in model C behave as the log of variables in model B.
In addition, in model B, variables have distinct steady-state values depending on the parameters of the model,
  while in model C, the steady-state of all variables is zero.
If, however, we would add hat variables to model B,
  then the hat variables would be equivalent between model variants.

\item The exponential transform enables one to write variables in terms of log deviations from their steady-state:
\begin{align*}
x_t = e^{log(x_t)} = e^{log(x_t) - log(x) + log(x)} = x e^{log(x_t)-log(x)} = x e^{\hat{x}_t}
\end{align*}
It is very useful to do this transform, when you want to log-linearize your model,
  i.e.\ do a first-order Taylor approximation of the model equations with respect to logged variables (HAT VARIABLES $\hat{x}_t$).\footnote{
Note that there are shortcuts to do a log-linearization without doing the exp transform for purely multiplicative or additive equations.
However, the exp transform always works, particularly for more difficult equations, so no need for learning the shortcuts.
}
\begin{itemize}    
    \item Lets use this on equation \eqref{eq:AS_B1}:
    \begin{align*}
    1 = \beta E_t \left[ \left(\frac{c e^{\hat{c}_{t+1}}}{c e^{\hat{c}_t}}\right)^{-\tau} \frac{1}{\gamma z e^{\hat{z}_{t+1}}} \frac{R e^{\hat{R}_t}}{\pi e^{\hat{\pi}_{t+1}}} \right]
    \end{align*}
    Note that equation \eqref{eq:AS_B1} in steady-state is equal to $1 = \beta \frac{1}{\gamma z}\frac{R}{\pi}$.
    So we can simplify the previous equation to get equation \eqref{eq:AS_C1}:
    \begin{align*}
    1 &= \mathbb{E}_t \left[e^{-\tau \hat{c}_{t+1} + \tau \hat{c}_{t} + \hat{R}_{t} - \hat{z}_{t+1} - \hat{\pi}_{t+1} }\right]
    \end{align*}    
    \item Log-Linearization: the first-order Taylor expansion of equation \eqref{eq:AS_C1} with respect to the hat variables is:
    \begin{align*}
    1 = e^0 + e^0 E_t \left(-\tau \hat{c}_{t+1} + \tau \hat{c}_{t} + \hat{R}_{t} - \hat{z}_{t+1} - \hat{\pi}_{t+1}  \right)\\
    \Leftrightarrow
    0 = -\tau E_t \hat{c}_{t+1} + \tau \hat{c}_{t} + \hat{R}_{t} - E_t\hat{z}_{t+1} - E_t \hat{\pi}_{t+1}
    \end{align*}
    The first-order Taylor expansion of equation \eqref{eq:AS_C3} with respect to the hat variables is:
    \begin{align*}
    e^0 + e^0\left(\hat{c}_t - \hat{y}_t\right) = e^0 - \frac{\phi \pi^2 g}{2} (e^0-1)^2 + e^0(-\hat{g}_t) - \frac{\phi \pi^2 g}{2} 2 \left(e^0-1\right)(\hat{\pi}_t)\\
    \Leftrightarrow \hat{c}_t = \hat{y}_t -\hat{g}_t
    \end{align*}
    Combining yields equation \eqref{eq:AS_D1}:
    \begin{align*}
    \hat{y}_{t} &= \mathbb{E}_t \hat{y}_{t+1} + \hat{g}_{t} - E_t\hat{g}_{t+1} - \frac{1}{\tau} \left(\hat{R}_{t}- \mathbb{E}_t \hat{\pi}_{t+1} - \mathbb{E}_t \hat{z}_{t+1}\right)
    \end{align*}
\end{itemize}

\item Note that we also added the hat variables to make the equivalence between model variants even more evident:
\lstinputlisting[style=Matlab-editor,basicstyle=\mlttfamily\scriptsize,title=\lstname]{progs/dynare/an_schorfheide_nonlinear.mod}
\begin{verbatim}
THEORETICAL MOMENTS
VARIABLE         MEAN  STD. DEV.   VARIANCE
c              0.9487     0.0058     0.0000
z              1.0000     0.0069     0.0000
pie            1.0080     0.0070     0.0000
R              1.0156     0.0083     0.0001
Rstar          1.0156     0.0113     0.0001
y              1.1161     0.0225     0.0005
ystar          1.1161     0.0214     0.0005
g              1.1765     0.0226     0.0005
chat           0.0000     0.0061     0.0000
zhat           0.0000     0.0069     0.0000
piehat         0.0000     0.0069     0.0000
Rhat           0.0000     0.0082     0.0001
yhat           0.0000     0.0202     0.0004
ghat           0.0000     0.0192     0.0004
YGR            0.5000     1.1045     1.2199
INFL           3.2000     2.7787     7.7214
INT            6.2000     3.2712    10.7006	
\end{verbatim}
Note that the moments and irfs are exactly the same between model variants B, C and D when you compare the hat variables and measurement equations.

\item 
\lstinputlisting[style=Matlab-editor,basicstyle=\mlttfamily\scriptsize,title=\lstname]{progs/dynare/an_schorfheide_nonlinear_exp.mod}
\begin{verbatim}
THEORETICAL MOMENTS
VARIABLE         MEAN  STD. DEV.   VARIANCE
chat           0.0000     0.0061     0.0000
zhat           0.0000     0.0069     0.0000
piehat         0.0000     0.0069     0.0000
Rhat           0.0000     0.0082     0.0001
yhat           0.0000     0.0202     0.0004
ghat           0.0000     0.0192     0.0004
YGR            0.5000     1.1045     1.2199
INFL           3.2000     2.7787     7.7214
INT            6.2000     3.2712    10.7006	
\end{verbatim}
Note that the moments and irfs are exactly the same between model variants B, C and D when you compare the hat variables and measurement equations.

\item 
\lstinputlisting[style=Matlab-editor,basicstyle=\mlttfamily\scriptsize,title=\lstname]{progs/dynare/an_schorfheide_loglinear.mod}
\begin{verbatim}
THEORETICAL MOMENTS
VARIABLE         MEAN  STD. DEV.   VARIANCE
chat           0.0000     0.0061     0.0000
zhat           0.0000     0.0069     0.0000
piehat         0.0000     0.0069     0.0000
Rhat           0.0000     0.0082     0.0001
yhat           0.0000     0.0202     0.0004
ghat           0.0000     0.0192     0.0004
YGR            0.5000     1.1045     1.2199
INFL           3.2000     2.7787     7.7214
INT            6.2000     3.2712    10.7006	
\end{verbatim}
Note that the moments and irfs are exactly the same between model variants B, C and D when you compare the hat variables and measurement equations.

\item Note that all model variants yield the same model dynamics evident in the same moments and impulse response functions for the hat variables or the observable variables.
Therefore, one can simply add auxiliary variables for the log or the percentage deviation of a variable to a nonlinear mod file.
So there is no need to do a log-linearization on a model per se.
Moreover, log-linearizing the model equations by hand is often tedious and very error-prone.
By defining logged variables in the original nonlinear model equations and using first-order perturbation techniques,
  effectively Dynare does the linearization for you.
So in most cases, it is actually better to focus on the nonlinear stationarized model equations.
Also, you cannot do higher-order approximations on the log-linearized system.

On the other hand, log-linearized expressions often provide clear analytic intuition
  and reduce the model size which might be beneficial for e.g. advanced estimation techniques
  (for standard ones this is not really necessary).
\end{enumerate}
\fi
\newpage
\end{solution}