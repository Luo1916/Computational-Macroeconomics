\section[Symbolic Math Toolbox in MATLAB]{Symbolic Math Toolbox in MATLAB\label{ex:MatlabSymbolicToolbox}}
MATLAB's Symbolic Math Toolbox provides a set of functions for solving, plotting, and manipulating symbolic math equations.
It enables you to perform symbolic computations by defining a special data type — symbolic objects.
Functions are called using the familiar MATLAB syntax
  and are available for integration, differentiation, simplification, equation solving, and other mathematical tasks.

\subsection*{Creating symbolic numbers, matrices, and variables}
Symbolic numbers are exact representations (like we do on paper).
Calculations on symbolic numbers and variables are exact,
  unlike floating-point numbers which are only accurate to machine precision.\footnote{%
  Granted MATLAB also offers variable-precision floating-point arithmetic (VPA).
}

In MATLAB you can create symbolic objects by using either \texttt{syms} or \texttt{sym}.
The \texttt{syms} command is actually shorthand for the \texttt{sym} syntax,
  but the two functions handle assumptions differently.
For example \texttt{syms x} creates a symbolic variable x in the MATLAB workspace with the value x,
  whereas \texttt{z = sym{('y')}} creates a symbolic variable z with the value y.

\begin{enumerate}

\item
Create the symbolic number \(sym(1/3)\) and compare it to the floating number \(1/3\).
How does MATLAB display these numbers in the terminal?

\item
Demonstrate the exactness of symbolic numbers by computing \(\sin(\pi)\) symbolically and numerically.

\item
You can create multiple variables in one command with \texttt{syms x y z}.
Create the variables x, y, and z.

\item
If you want to create an array of numbered symbolic variables, the \texttt{syms} syntax is inconvenient.
Use \texttt{sym{('a',[1 20])}} instead to create a row vector containing the symbolic variables
  \(a_{1},\ldots ,a_{20}\) and assign it to the MATLAB variable \(A\).

\item
By combining \texttt{sym} and \texttt{syms}, you can create many symbolic variables
  with corresponding variables name in the MATLAB workspace.
Clear the workspace.
Create the symbolic variables \(a_{1},\ldots ,a_{10}\)
  and assign them the MATLAB variable names \(a_{1},\ldots ,a_{10}\), respectively,
  by running \texttt{syms{(sym{('a', [1 10])})}}.
Have a look in the workspace which variables are created.

\item
Create the following symbolic matrix
\begin{align*}
A = \begin{pmatrix} a & b & c\\c & a & b\\b & c & a \end{pmatrix}
\end{align*}
Check if the sum of the elements of the first row equals the sum of the elements of the second column.

\item
The \texttt{sym} function also lets you define a symbolic matrix or vector without having to define its elements in advance.
In this case, the \texttt{sym} function generates the elements of a symbolic matrix at the same time that it creates a matrix.
The function presents all generated elements using the same form:
  the base (which must be a valid variable name), a row index, and a column index.
Run the following code
\begin{lstlisting}[style=Matlab-editor,basicstyle=\mlttfamily\scriptsize]
clearvars
A = sym('a', [2 4])
B = sym('b_%d_%d', [2 4])
\end{lstlisting}
  and confirm that you understood the syntax by looking at the created variables.

\end{enumerate}

\subsection*{Perform symbolic computations}
With the Symbolic Math Toolbox software, you can also perform symbolic computations like derivatives, integrals or solving equations.

\begin{enumerate}[resume]

\item
Have a look at the output of the following commands:
\begin{lstlisting}[style=Matlab-editor,basicstyle=\mlttfamily\scriptsize]
clearvars
syms a b c x;
f = a*x^2 + b*x + c
g(x) = a*x^2 + b*x + c
g(4)
solve(g)
solve(f==2,a)
\end{lstlisting}
What does \texttt{solve{(g)}} and \texttt{solve{(f,a)}} do?

\item
To differentiate a symbolic expression, use the \texttt{diff} or \texttt{jacobian} commands.
Take the first derivative of \(f\) with respect to \(x\) and \(a\), respectively:
\begin{lstlisting}[style=Matlab-editor,basicstyle=\mlttfamily\scriptsize]
clearvars
syms a b c x;
f = a*x^2 + b*x + c
diff(f,x)
diff(f,a)
diff(f,b)
diff(f,c)
diff(f,x,a)
jacobian(f,[a;b;c;x])
\end{lstlisting}

\item
To compute the integral use the \texttt{int} function.
Compute \(\int_{-\infty}^\infty a x^2 + b x + c dx\) and \(\int_{-2}^1 a x^2 + b x + c dx\).

\end{enumerate}

\subsection*{Symbolic simplifications}
The Symbolic Math Toolbox provides a set of simplification functions
  allowing you to manipulate the output of a symbolic expression, e.g.\
  \texttt{simplify}, \texttt{expand}, \texttt{factor}, \texttt{horner} or \texttt{subs}.
Symbolic simplification is not always straightforward
  as different problems require different forms of the same mathematical expression.

\begin{enumerate}[resume]

\item
Define the following polynomial of the golden ratio:
\begin{lstlisting}[style=Matlab-editor,basicstyle=\mlttfamily\scriptsize]
phi = (1 + sqrt(sym(5)))/2;
golden_ratio = phi^2 - phi - 1
\end{lstlisting}
Now run \texttt{simplify{(golden\_ratio)}}.

\item
To show the order of a polynomial or to symbolically differentiate or integrate a polynomial
  the standard polynomial form with all the parentheses multiplied out
  and all the similar terms summed up is useful.
You can do this by using \texttt{expand}:
\begin{lstlisting}[style=Matlab-editor,basicstyle=\mlttfamily\scriptsize]
clearvars
syms x
f = (x^2-1)*(x^4+x^3+x^2+x+1)*(x^4-x^3+x^2-x+1);
disp(f);
expand(f)
\end{lstlisting}

\item
To show the polynomial roots a factor simplification is useful:
\begin{lstlisting}[style=Matlab-editor,basicstyle=\mlttfamily\scriptsize]
clearvars
syms x
g = x^3 + 6*x^2 + 11*x + 6;
disp(g)
factor(g)      
\end{lstlisting}

\item
The nested (Horner) representation of a polynomial is often most efficient for numerical evaluations:
\begin{lstlisting}[style=Matlab-editor,basicstyle=\mlttfamily\scriptsize]
clearvars
syms x
h = x^5 + x^4 + x^3 + x^2 + x;
disp(h)
horner(h)
\end{lstlisting}

\item
You can substitute a symbolic variable with another symbolic variable
  or a numeric value by using the subs function.
When your expression contains more than one variable,
  you can specify the variable for which you want to make the substitution. 
For example:
\begin{lstlisting}[style=Matlab-editor,basicstyle=\mlttfamily\scriptsize]
clearvars
syms x y
f = x^2*log(y) + 5*x*sqrt(y);
disp(f)
subs(f, x, 3)
subs(f, x, y)    
\end{lstlisting}
You can also substitute a set of elements in a symbolic matrix:
\begin{lstlisting}[style=Matlab-editor,basicstyle=\mlttfamily\scriptsize]
syms a b c
alpha = sym('alpha');
beta = sym('beta');
A = [a b c; c a b; b c a];
disp(A);
A(2,1) = beta;
subs(A,b,alpha);
B = subs(A,b,alpha);
disp(A);
disp(B)       
\end{lstlisting}
\end{enumerate}
Try out these codes.

\subsection*{Writing out symbolic expressions to script files for numerical evaluation}
Often we want to do some symbolic manipulations and then evaluate those expressions numerically.
Evaluating symbolic variables, however, is computationally expensive
  in terms of memory allocation and speed of numerical evaluation.
One can write out the symbolic expressions to script files and evaluate these,
  because evaluating functions is computationally very cheap and fast.
Consider the following example, where we want to compute the Jacobian of \(f\):
\begin{lstlisting}[style=Matlab-editor,basicstyle=\mlttfamily\scriptsize]
clearvars
var_names = ["x";"y"];
param_names = ["ALPHA";"BETA"];
syms(sym(var_names));
syms(sym(param_names));
f(1,1) = x^ALPHA*log(y) + BETA*x*sqrt(y);
f(2,1) = x^2-y + BETA/ALPHA*sqrt(x^3)*(y/2);
f(3,1) = x^ALPHA + BETA*x;
f(4,1) = log(y) + BETA*sqrt(y);
df = jacobian(f,[x;y]);
\end{lstlisting}

\newpage

\begin{enumerate}[resume]

\item
Have a look at the help file and the examples for \texttt{fprintf}
  which is able to print formatted data to a text file.

\item
Run the following commands:
\begin{lstlisting}[style=Matlab-editor,basicstyle=\mlttfamily\scriptsize]
NameOfFunction = 'num_df';
NameOfFile = strcat(NameOfFunction,'.m');
NameOfOutput = 'df';
if exist(NameOfFile,'file') > 0
    delete(NameOfFile);
end
fileID = fopen(NameOfFile,'w');
fprintf(fileID,'function %s = %s(%s)\n',...
          NameOfOutput,NameOfFunction,strjoin([var_names;param_names],','));    
fprintf(fileID,'%s = zeros(%d, %d);\n',NameOfOutput,size(df,1),size(df,2));
[nonzero_row,nonzero_col,nonzero_vals] = find(df);
for j = 1:size(nonzero_vals,1)
    fprintf(fileID,'%s(%d,%d) = %s;\n',... 
          NameOfOutput,nonzero_row(j),nonzero_col(j),char(nonzero_vals(j)));
end
fprintf(fileID,'\nend %% function end \n');
fclose(fileID);
\end{lstlisting}
Note that a new file called \texttt{num{\_}df.m} is created in your working directory.
Open it and try to understand what the commands actually do.

\item
Now let's compare the runtime of computing \(df\) a 100 times for random data by
  (i) substituting the symbolic expressions or (ii) evaluating the script files:
\begin{lstlisting}[style=Matlab-editor,basicstyle=\mlttfamily\scriptsize]
rng(123); % get same random numbers by setting seed
tic;
for j=1:100
    num_df(randn(),randn(),rand(),rand());
end
toc
  
rng(123); % get same random numbers by setting seed
tic;
for j=1:100
    double(subs(df,[x y ALPHA BETA],[randn(),randn(),rand(),rand()]));
end
toc
\end{lstlisting}

\end{enumerate}

\paragraph{Readings}

{\scriptsize
\begin{itemize}
\item \url{https://mathworks.com/help/symbolic/getting-started-with-symbolic-math-toolbox.html}
\item \url{https://mathworks.com/help/symbolic/symbolic-computations-in-matlab.html}
\end{itemize}
}

\begin{solution}\textbf{Solution to \nameref{ex:MatlabSymbolicToolbox}}
\ifDisplaySolutions%
\lstinputlisting[style=Matlab-editor,basicstyle=\mlttfamily\scriptsize,title=\lstname]{progs/matlab/symbolicToolboxIllustation.m}
\fi
\newpage
\end{solution}