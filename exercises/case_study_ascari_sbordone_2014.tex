\section[Case Study Ascari and Sbordone (2014): The Macroeconomics of Trend Inflation]{Case Study Ascari and Sbordone (2014): The Macroeconomics of Trend Inflation\label{ex:CaseStudy.Ascari.Sbordone.2014}}

\begin{enumerate}
\item Read the paper by \textcite{Ascari.Sbordone_2014_MacroeconomicsTrendInflation} and its Appendix.
How does the model differ from the New Keynesian model that we studied in the course so far?

\item Consider the following Dynare files given in the appendix:
\begin{itemize}
  \item \texttt{ascari\_sbordone\_2014\_common.mod}
  \item \texttt{ascari\_sbordone\_2014\_calib\_common.inc}
\end{itemize}
Explain what each file is doing.

\item Replicate figure 14 of the paper.
Set initially \(\varphi = 1\), \(\phi_\pi=2\), \(\phi_y = 0.5/4\) and \(\rho_i=0.8\).

\item Replicate figure 13 of the paper.
Set initially \(\varphi = 0\), \(\phi_\pi=1.5\), \(\phi_y = 0.5/4\) and \(\rho_i=0\).

\item Replicate the business cycle moments reported on page 717.
Set initially \(\varphi = 1\), \(\rho_a=0.95\), \(\rho_i=0\) \(\phi_\pi=1.5\), and \(\phi_y = 0.5/4\).
Note that for the business cycle moments, the technology shock standard error is actually \(0.45\).

\item Replicate figure 7 of the paper.
Set initially \(\varphi = 1\), \(\phi_\pi=2\), \(\phi_y = 0.5/4\) and \(\rho_i=0.8\).

\item Replicate figure 8 of the paper.
Set initially \(\varphi = 1\), \(\phi_\pi=2\), \(\phi_y = 0.5/4\) and \(\rho_i=0.8\).

\item Replicate figure 11 of the paper.
Set initially \(\varphi = 1\), \(\phi_\pi=2\), \(\phi_y = 0.5/4\) and \(\rho_i=0.8\).
Note that as described in footnote 54, the determinacy region in Figure 11 is actually the \enquote{determinacy and stability region},
  i.e.\ it does not distinguish whether the Blanchard-Kahn conditions fail because of too many unstable roots (instability, info==3)
  or because of too few unstable roots (indeterminacy, info==4).

\end{enumerate}

\paragraph{Hints}

\begin{itemize}

\item The annual inflation target is disaggregated to quarterly figures using the geometric mean.
Simply dividing by 4 results in small numerical differences.

\item The labor disutility parameter, which is unspecified in the paper is set so that labor is 1/3 in the benchmark case.

\item The technology parameter, which is also left unspecified in the published version, is set to 1 in steady-state.
  
\end{itemize}


\begin{solution}\textbf{Solution to \nameref{ex:CaseStudy.Ascari.Sbordone.2014}}
\ifDisplaySolutions%
\input{exercises/case_study_ascari_sbordone_2014_solution.tex}
\fi
\newpage
\end{solution}