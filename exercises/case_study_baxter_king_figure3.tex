\section[Case Study Baxter and King (1993, AER): Macroeconomic Effects Of A Four-Year War]{Case Study Baxter and King (1993, AER): Macroeconomic Effects Of A Four-Year War\label{ex:CaseStudy.BaxterKing.Figure3}}

\begin{enumerate}
\item
Read the paper by \textcite{Baxter.King_1993_FiscalPolicyGeneral}
  and focus particularly on the model equations and analysis done in section IV.A.

\item
Write a Dynare mod file for the model and calibrate it according to
  the \emph{Benchmark Model with Basic Government Purchases} given in Table 1.
Compute the steady-state using an \texttt{initval} block.

\item
Replicate Figure 3.

\item
Do you think this simulation scenario reflects the essence of a war shock?

\end{enumerate}

\paragraph{Notes and hints}

\begin{itemize}
\item
According to footnote 3, the quantitative analysis corresponds to a model with labor-augmenting technical progress.
Calibrate \(\gamma_X\) such that the economy grows at 1.6\% per annum,
  a typical value used in e.g.\ \textcite{King.Plosser.Rebelo_1988_ProductionGrowthBusiness}.

\item
Pay attention to the different scales on the y-axes,
  i.e.\ commodity units, percent, and basis points.

\item
Skip the term structure variable in panel \emph{C.},
  as we will re-visit the term structure in RBC models later on.

\item
Use a perfect foresight simulation for computing the transition path
  from the initial steady-state (specified by an \texttt{initval} block)
  subject to a sequence of four basic government purchases shocks
  (specified by a \texttt{shocks} block).

\end{itemize}

\begin{solution}\textbf{Solution to \nameref{ex:CaseStudy.BaxterKing.Figure3}}
\ifDisplaySolutions%
\begin{enumerate}

\item[2.~and 3.]
The mod file might look like this:
\lstinputlisting[style=Matlab-editor,basicstyle=\mlttfamily\scriptsize,title=\lstname]{progs/replications/Baxter_King_1993/Baxter_King_1993_figure_3.mod}

\item[4.]
No, for many reasons. To name just two:

\begin{itemize}

\item
A war is a large shock with unknown duration.
Perfect foresight simulations do not capture this as in period 1 everything is revealed to the agents.
A solution to this would be a perfect-foresight simulation with expectation errors (sometimes called MIT-shock).
To this end, one simulates the sequence of shocks
  but alters the information set of the agents in each period
  before the onset of another shock.
Algorithmically, we repeat the simulation cycle four times
  and then combine the simulations,
  utilizing the first one for periods \(1\) to \(2\),
  the second one for \(2\) to \(3\), the third one for \(3\) to \(4\) \ldots,
  and the fourth one for \(4\) to \(T\).
In Dynare 6.0 there is a new command for such a simulation called
  \texttt{perfect{\_}foresight{\_}with{\_}expectation{\_}errors}.
For older versions of Dynare this type of simulation can be done
  by running a sequence of perfect foresight simulations
  while adjusting initial conditions manually.
Intuitively, this captures the belief that guides agents during times of war;
  namely, that the conflict will last only one year
  (the surprise shock on impact) and will not occur again.
However, this belief is challenged when the war continues beyond the expected duration,
  and agents are surprised by the ongoing conflict
  (restart the simulation with a new surprise shock in the next period).
Despite this surprise, the belief in a one-year duration persists,
  leading to ongoing cycles of surprise shocks (and expectation errors).

\item
A war shock has many channels;
  solely relying on a demand-side effect via government spending
  reflects a very US-specific view.
There is research that war shocks are rather dominated by supply-side effects
  and also accompanied by different policy stances during war times.
One could argue that the current simulation is for a neighboring country,
  as the US has never really been a war site itself
  (except for comparably minor battles of Pearl Harbor and the Aleutian islands
  during World War II).
\end{itemize}

\end{enumerate}

\fi
\newpage
\end{solution}