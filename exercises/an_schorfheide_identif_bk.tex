\section[Identification and Sensitivity Analysis]{Identification and Sensitivity Analysis\label{ex:AnScho2007_identif_bk}}
Consider the New Keynesian model of \textcite{An.Schorfheide_2007_BayesianAnalysisDSGE}.
The model equations are given by 
\begin{align}
1 &= \beta \mathbb{E}_t\left[\left(\frac{c_{t+1}}{c_t}\right)^{-\tau} \frac{1}{\gamma z_{t+1}} \frac{R_t}{\pi_{t+1}}\right] \label{eq:AS_Euler}
\\
1 &= \phi \left(\pi_t - \pi\right) \left[\left(1-\frac{1}{2\nu}\right)\pi_t + \frac{\pi}{2\nu}\right] - \phi \beta \mathbb{E}_t \left[\left(\frac{c_{t+1}}{c_t}\right)^{-\tau} \frac{y_{t+1}}{y_t} \left(\pi_{t+1} - \pi \right) \pi_{t+1}\right] + \frac{1}{\nu}\left[1-c_t^{\tau}\right] \label{eq:AS_PC}
\\
y_t &= c_t + \frac{g_t-1}{g_t} y_t + \frac{\phi}{2} \left({\pi_t - \pi}\right)^2 y_t \label{eq:AS_Market}
\\
R_t &= {R_t^{*}}^{1-\rho_R} R_{t-1}^{\rho_R} e^{\epsilon_{R,t}} \label{eq:AS_Taylor1}
\\
R_t^* &= R \left(\frac{\pi_t}{\bar{\pi}}\right)^{\psi_1} \left(\frac{y_t}{y_t^*}\right)^{\psi_2} \label{eq:AS_Taylor2}
\\
\ln(z_t) &= \rho_z \ln(z_{t-1}) + \epsilon_{z,t} \label{eq:AS_tfpgrowth}
\\
\ln(g_t) &= (1-\rho_g)\ln\left(\bar{g}\right) + \rho_g \ln(g_{t-1}) + \epsilon_{g,t} \label{eq:AS_g}
\\
y_t^* &= (1-\nu)^{\frac{1}{\tau}} g_t \label{eq:AS_flexy}
\end{align}
where $\epsilon_{R,t}$, $\epsilon_{g,t}$ and $\epsilon_{z,t}$ are iid normally distributed with a mean of zeros and standard errors equal to $\sigma_r$, $\sigma_g$, $\sigma_z$, respectively.
Moreover, we have the following auxiliary parameter relationships:
\begin{align*}
\gamma = 1+\frac{\gamma^{Q}}{100}, \qquad
\beta = \frac{1}{1+r^{A}/400}, \qquad
\bar{\pi} = 1+\frac{\pi^{A}}{400}, \qquad
\phi = \tau \frac{1-\nu}{\nu \bar{\pi}^{2}\kappa}
\end{align*}
Note that $y_t^*$ is an endogenous model variable, whereas $\bar{\pi}$ and $\bar{g}$ are parameters and $R$ denotes the steady-state of $R_t$.

\paragraph{Parametrization}
If not otherwise stated, use the following parameter values for the exercises:
\begin{align*}
&\tau=2,\quad
\kappa=0.33,\quad
\nu=0.1,\quad
\rho_g=0.95,\quad
\rho_z=0.9,\quad	
r^{A}=1,\quad
\\
&\pi^{A}=3.2,\quad
\gamma^{Q}=0.5,\quad
\bar{g}=1/0.85,\quad
\sigma_z = \sqrt{0.9},\quad
\sigma_g=\sqrt{0.36}
  \end{align*}
We will consider two different parametrizations of the monetary policy rule $(\psi_1,\psi_2,\rho_R,\sigma_r)$, one is called \textbf{hawkish}:
\begin{align*}
&\psi_1=1.5,&& \psi_2=0.125,&& \rho_R = 0.75,&& \sigma_r = \sqrt{0.4}	\\
\end{align*}
and the other one \textbf{dovish}:
\begin{align*}
&\psi_1=1.043195,&& \psi_2=0.918417,&& \rho_R = 0.792647,&& \sigma_r = \sqrt{0.446783}	
\end{align*}

\newpage

\subsection*{Exercises}

\begin{enumerate}

\item Provide intuition behind each equation of the model.
How are nominal price rigidities modelled?\\\emph{Hint: Have a look at the relevant parts of the underlying paper.}

\item Write a Dynare mod file for this model. Derive the steady-state of all model variables analytically and include these in a \texttt{steady\_state\_model} block.

\item Explain how theoretical moments are computed based on the first-order policy function.

\item Investigate the steady-state properties, theoretical moments, and impulse response functions of the two different parametrizations (dovish vs. hawkish) 
  by running the \texttt{stoch\_simul(order=1)} command.
\\
\emph{Hint: Look at Dynare's output in the command window for the Steady-State values and Theoretical Moments.}

\item The determinacy region is defined as the parameter space for which the Blanchard \& Khan order and rank conditions are full-filled.
Dynare's sensitivity toolbox provides means to get a graphical representation of this region
  by drawing randomly from the parameter space and checking the order and rank conditions.
Add the following code snippet to your mod file:
\begin{verbatim}
%---------------
%specify parameters for which to map sensitivity
estimated_params;
PSI1, uniform_pdf,,, 0,6; %draw uniformly from 0 to 6
PSI2, uniform_pdf,,,-1,6; %draw uniformly from -1 to 6
RHOR, uniform_pdf,,,-1,1; %draw uniformly from -1 to 1
end;
varobs y pie R;
dynare_sensitivity(prior_range=0,stab=1,nsam=2000);		
%---------------
\end{verbatim}	
  which randomly draws $\psi_1 \in [0,6]$, $\psi_2 \in [-1,6]$ and $\rho_R \in [-1,1]$,
  while keeping all other parameter values at their calibrated values.
Focus on the the plots and explain these.
What does this imply for the ability of monetary policy to anchor inflation expectations?
\end{enumerate}

\paragraph{Readings}
\begin{itemize}
	\item \textcite{An.Schorfheide_2007_BayesianAnalysisDSGE}
	\item \textcite{Ivashchenko.Mutschler_2020_EffectObservablesFunctional}
\end{itemize}

\begin{solution}\textbf{Solution to \nameref{ex:AnScho2007_identif_bk}}
\ifDisplaySolutions%
\input{exercises/an_schorfheide_identif_bk_solution.tex}
\fi
\newpage
\end{solution}