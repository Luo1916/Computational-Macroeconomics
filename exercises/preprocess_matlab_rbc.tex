\section[RBC model: preprocessing in MATLAB]{RBC model: preprocessing in MATLAB\label{ex:PreprocessMATLABRBC}}
Consider the basic Real Business Cycle (RBC) model with leisure and log utility function.
The dynamic model equations are given in Table~\ref{tbl:RBC.Model},
  the variable description in Table~\ref{tbl:RBC.Variables},
  and the description and calibration of parameters in Table~\ref{tbl:RBC.Parameters}.

{\footnotesize
\begin{center}
\begin{longtable}{lc}
\caption{Dynamic Model Equations\label{tbl:RBC.Model}}\\%
\toprule%
\multicolumn{1}{c}{\textbf{Equation}} &
\multicolumn{1}{c}{\textbf{Description}}\\%
\midrule\midrule%
\endfirsthead%
\multicolumn{2}{c}{{\tablename} \thetable{} {--} Continued}\\%
\midrule%
\multicolumn{1}{c}{\textbf{Variable}} &
\multicolumn{1}{c}{\textbf{Description}}\\%
\midrule\midrule%
\endhead%
\({{\gamma}}\, {{c}}_{t}^{-1}={{\gamma}}\, {{\beta}}\, {{c}}_{t+1}^{-1}\, \left(1-{{\delta}}+{{r}}_{t+1}\right)\) & intertemporal optimality (Euler)
\\
\({{w}}_{t}=-\frac{-{{\psi}}\, {\left(1-{{n}}_{t}\right)}^{-1}}{{{\gamma}}\, {{c}}_{t}^{-1}}\) & labor supply
\\
\({{k}}_{t}=\left(1-{{\delta}}\right)\, {{k}}_{t-1}+{{i}}_{t}\) & capital accumulation
\\
\({{y}}_{t}={{c}}_{t}+{{i}}_{t}\) & market clearing
\\
\({{y}}_{t}={{a}}_{t}\, {{k}}_{t-1}^{{{\alpha}}}\, {{n}}_{t}^{1-{{\alpha}}}\) & production function
\\
\({{w}}_{t}=\left(1-{{\alpha}}\right)\frac{{{y}}_{t}}{{{n}}_{t}}\) & labor demand
\\
\({{r}}_{t}={{\alpha}}\frac{{{y}}_{t}}{{{k}}_{t-1}}\) & capital demand
\\
\(\log\left({{a}}_{t}\right)= (1-\rho) \log{(a)} + {{\rho}}\, \log\left({{a}}_{t-1}\right)+{{\varepsilon}}_{t}\) & total factor productivity
\\
\bottomrule%
\end{longtable}
\end{center}
    
\begin{minipage}{.3\linewidth}
\begin{center}
\begin{longtable}{cc}
\caption{Variables\label{tbl:RBC.Variables}}\\%
\toprule
\multicolumn{1}{c}{\textbf{Variable}} &
\multicolumn{1}{c}{\textbf{Description}}\\%
\midrule\midrule%
\endfirsthead%
\multicolumn{2}{c}{{\tablename} \thetable{} {--} Continued}\\%
\midrule%
\multicolumn{1}{c}{\textbf{Variable}} &
\multicolumn{1}{c}{\textbf{Description}}\\%
\midrule\midrule%
\endhead%
\({y}\) & output\\
\({c}\) & consumption\\
\({k}\) & capital\\
\({n}\) & labor\\
\({a}\) & productivity\\
\({r}\) & interest rate\\
\({w}\) & wage\\
\({i}\) & investment\\
\({\varepsilon}\) & TFP shock\\
\bottomrule%
\end{longtable}
\end{center}
\end{minipage}%
\begin{minipage}{.65\linewidth}
\begin{center}
\begin{longtable}{ccc}
\caption{Parameter Values\label{tbl:RBC.Parameters}}\\%
\toprule%
\multicolumn{1}{c}{\textbf{Parameter}} &
\multicolumn{1}{c}{\textbf{Value}} &
\multicolumn{1}{c}{\textbf{Description}}\\%
\midrule%
\endfirsthead%
\multicolumn{3}{c}{{\tablename} \thetable{} {--} Continued}\\%
\midrule%
\multicolumn{1}{c}{\textbf{Parameter}} &
\multicolumn{1}{c}{\textbf{Value}} &
\multicolumn{1}{c}{\textbf{Description}}\\%
\midrule%
\endhead%
\({\beta}\)  & 0.990 & discount factor\\
\({\delta}\) & 0.025 & depreciation rate\\
\({\gamma}\) & 1.000 & consumption utility weight\\
\({\psi}\)   & 1.600 & labor disutility weight\\
\({\alpha}\) & 0.350 & elasticity of capital\\
\({\rho}\) 	 & 0.900 & persistence technology\\
\(A\)        & 10    & normalization\\
\bottomrule%
\end{longtable}
\end{center}
\end{minipage}
}%

\begin{enumerate}

\item
Create a Dynare mod file for this model and compute the steady-state.

\item
Notice that Dynare's preprocessor creates a folder with a + and the name of your mod file.
Inside the folder you can find different files.
Briefly explain what the script files \texttt{driver.m},
  \texttt{dynamic\_resid.m}, \texttt{dynamic\_g1.m},
  \texttt{static\_resid.m} and \texttt{static\_g1.m} do.

\item
Briefly explain Dynare's \texttt{M\_.lead\_lag\_incidence} matrix.

\item
Use MATLAB's Symbolic Math Toolbox to process the same model without Dynare.
To this end:

\begin{enumerate}

  \item
  Create the string arrays
  \texttt{endo\_names}, \texttt{exo\_names} and \texttt{param\_names}
  that contain the names of the endogenous and exogenous variables
    as well as the parameters.
  
  \item
  Store the length of the vectors to variables called
    \texttt{endo\_names\_nbr}, \texttt{exo\_names\_nbr}, and \texttt{param\_names\_nbr}.

  \item
  Create symbolic variables in the workspace with a
    \texttt{\_back}, \texttt{\_curr}, \texttt{\_fwrd}, and \texttt{\_stst} suffix,
	using the names in \texttt{endo\_names} and the \texttt{syms} command.
  That is, \texttt{a\_back} should be a symbolic variable denoting \(a_{t-1}\),
	\texttt{a\_curr} should be a symbolic variable denoting \(a_{t}\),
	\texttt{a\_fwrd} should be a symbolic variable denoting \(a_{t+1}\),
	and \texttt{a\_stst} should be a symbolic variable denoting the steady-state \(a\).

  \item
  Create symbolic variables using the names in \texttt{exo\_names} and the \texttt{syms} command.

  \item
  Create symbolic variables using the names in \texttt{param\_names} and the \texttt{syms} command.

  \item
  Create a symbolic vector \texttt{dynamic\_eqs}
    with the dynamic model equations given in Table~\ref{tbl:RBC.Model}
	(put everything on the left-hand side).
  Use the just declared convention for symbolic variables and parameters.

  \item
  Create the \texttt{lead\_lag\_incidence} matrix and use it to distinguish the following types of variables:
  \begin{itemize}
	\item static variables, which appear only at \(t\), but neither at \(t-1\) nor at \(t+1\).
	\item predetermined variables, which appear only at \(t-1\), possibly at \(t\), but not at \(t+1\).
	\item forward variables, which appear only at \(t+1\), possibly at \(t\), but not at \(t-1\).
	\item mixed variables, which appear at \(t-1\) and \(t+1\), and possibly at \(t\).
	\item dynamic variables, which actually appear in the dynamic model equations
	\item exogenous variables
  \end{itemize}

  \item
  Compute the Jacobian of \texttt{dynamic\_eqs}
    with respect to symbolic dynamic variables and the symbolic exogenous variables.

  \item
  Compute the static model equations using the \texttt{subs} command
    to substitute the dynamic variables with their corresponding name
	without the \texttt{\_back}, \texttt{\_curr}, or \texttt{\_fwrd} suffix.
  Store the static model equations in the symbolic variable \texttt{static\_eqs}.

  \item
  Compute the Jacobian of \texttt{static\_eqs} with respect to symbolic endogenous variables.

  \item
  Write out the static and dynamic model equations and Jacobians to script files:
\begin{lstlisting}[style=Matlab-editor,basicstyle=\mlttfamily\scriptsize]
writeOut(static_eqs,'rbc_static_resid','residual',1,dynamic_names,endo_names,exo_names,param_names);
writeOut(static_g1,'rbc_static_g1','g1',1,dynamic_names,endo_names,exo_names,param_names);
writeOut(dynamic_eqs,'rbc_dynamic_resid','residual',0,dynamic_names,endo_names,exo_names,param_names);
writeOut(dynamic_g1,'rbc_dynamic_g1','g1',0,dynamic_names,endo_names,exo_names,param_names);
\end{lstlisting}
  The function \texttt{writeOut.m} is given in Appendix~\ref{app:writeOut}.

  \item
  Make the whole script a function called \texttt{preprocessingRBC.m}
    with an output variable MODEL,
	which is a structure containing information on the names
	and numbers of the endogenous and exogenous variables,
	the names and numbers of the parameters,
	and also the \texttt{lead\_lag\_incidence} matrix.

\end{enumerate}

\item Compare Dynare's preprocessing and the manual preprocessing done in MATLAB.\@
To this end:

\begin{enumerate}
  \item
  Run dynare on the mod file to create the preprocessing files
	and to compute the steady-state.
  Append the following code at the end of your mod file:
\begin{lstlisting}[style=Matlab-editor,basicstyle=\mlttfamily\scriptsize]
% create steady-state vectors
[I,~] = find(M_.lead_lag_incidence');
y_ss   = oo_.steady_state;      % steady-state of endogenous variables
zzz_ss = oo_.steady_state(I);   % steady-state of dynamic variables
ex_ss  = oo_.exo_steady_state'; % steady-state of exogenous variables

% evaluate dynamic model residuals and Jacobian at steady-state
[dynare_resid, dynare_g1] = feval([M_.fname,'.dynamic'], zzz_ss, ex_ss, M_.params, y_ss, 1);
% evaluate static model residuals and Jacobian at steady-state
[dynare_resid_static, dynare_g1_static] = feval([M_.fname,'.static'], y_ss, ex_ss, M_.params);
\end{lstlisting}

  \item
  Preprocess the model with MATLAB and evaluate the script files at the steady-state:

  \item
  Look at the differences of the computed residuals
    and Jacobians of both dynamic and static model equations
	between Dynare's and MATLAB's preprocessing.

\end{enumerate}

\end{enumerate}


\begin{solution}\textbf{Solution to \nameref{ex:PreprocessMATLABRBC}}
\ifDisplaySolutions%
\begin{enumerate}

\item
\lstinputlisting[style=Matlab-editor,basicstyle=\mlttfamily\scriptsize,title=\lstname]{progs/dynare/rbc.mod}

\item
Dynare's preprocessor creates the following script files:
\begin{itemize}
  \item
  \texttt{driver.m}: the preprocessor reads the text and information provided in the mod file and creates MATLAB code from this.
  The \texttt{driver.m} contains these transformations and it is pure MATLAB code,
    meaning that you can simply run it in MATLAB (e.g.\ hit the green run button).
  In this script all model information is stored into structures.
  Particularly, the global structures \texttt{M\_} (which contains model information),
    \texttt{options\_} (which contains default values for options),
	and \texttt{oo\_} (which contains results) are initialized
	(and some other structures as well depending on the commands you run in your mod file).

  \item
  \texttt{dynamic\_resid.m}: this is a script file with MATLAB code that evaluates the residuals of the dynamic model equations,
	i.e.\ what we get when we put all model equations on the right hand side
	and evaluate this system of equations for arbitrary values of the dynamic variables and the parameters.
  Note that the dynamic files distinguish that variables can appear at different timings \(t-1\), \(t\) or \(t+1\).

  \item
  \texttt{static\_resid.m}: this is a script file with MATLAB code that evaluates the residuals of the static model equations.
  The static model equations are given when we strip of the timings of the variables.
  The file then computes the residuals, i.e.\ when we put all model equations on the right hand side
	and evaluate the static system of equations for arbitrary values of the static variables and parameters.

  \item
  \texttt{dynamic\_g1.m}: this is a script file with MATLAB code that evaluates the Jacobian of the dynamic model equations
	with respect to the dynamic model variables and the exogenous shock.
  That is, let \(f(z)\) denote the model equations and \(z\) the vector of the dynamic variables and the exogenous variables,
	then \(g1\) computes \(\partial f(z) \ \partial z\).

  \item
  \texttt{static\_g1.m}: this is a script file with MATLAB code that evaluates the Jacobian of the static model equations
	with respect to the model variables.
  That is, let \(\bar{f}(y)\) denote the model equations and \(y\) the vector of the model variables,
	then this file computes \(\partial \bar{f}(y) \ \partial y\).
\end{itemize}

\item
The Jacobian \(g_1\) of the dynamic model is a key matrix for solving and simulating DSGE models.
It's row dimension is equal to the number of equations,
  and the columns correspond to which endogenous variable the derivative is taken with respect to.
First come the \(t-1\) variables (that actually appear in the model equations at \(t-1\)),
  then the \(t\) variables (that actually appear in the model equations at \(t\)),
  then the \(t+1\) variables (that actually appear in the model equations at \(t+1\)),
  and lastly the exogenous variables.
When computing the Jacobian of the dynamic model,
  the order of the endogenous variables in the columns is stored in \texttt{M\_.lead\_lag\_incidence}.
The rows of this matrix represent time periods: the first row denotes a lagged (time \(t-1\)) variable,
  the second row a contemporaneous (time \(t\)) variable,
  and the third row a leaded (time \(t+1\)) variable.
The columns of the matrix represent the endogenous variables in their order of declaration.
A zero in the matrix means that this endogenous does not appear in the model in this time period.
The value in the \texttt{M\_.lead\_lag\_incidence} matrix
  corresponds to the column of that variable in the Jacobian \(g_1\) of the dynamic model.

\item
A preprocessing script:
\lstinputlisting[style=Matlab-editor,basicstyle=\mlttfamily\scriptsize,title=\lstname]{progs/matlab/preprocessingRBC.m}

\item
A script for comparison:
\lstinputlisting[style=Matlab-editor,basicstyle=\mlttfamily\scriptsize,title=\lstname]{progs/matlab/preprocessingComparisonRBC.m}

\end{enumerate}
\fi
\newpage
\end{solution}