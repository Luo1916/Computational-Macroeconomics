\section[RBC model: steady-state in MATLAB]{RBC model: steady-state in MATLAB\label{ex:RBCModelSteadyMATLAB}}
Consider the basic Real Business Cycle (RBC) model with leisure and log utility function.
The dynamic model equations are given in Table~\ref{tbl:RBC.Model},
  the variable description in Table~\ref{tbl:RBC.Variables},
  and the description and calibration of parameters in Table~\ref{tbl:RBC.Parameters}.

\begin{enumerate}

\item
Create a Dynare mod file for this model and use a \texttt{steady\_state\_model} block to compute the steady-state.
Store the computed steady-state into a variable \texttt{stst\_dyn}.
Evaluate the static file at the steady-state and store your result into a variable \texttt{resid\_dyn}.
Compute the sum of squared residuals of \texttt{resid\_dyn} and store your result into a variable \texttt{ssq\_dyn}.

\item
Preprocess the model using the function \texttt{preprocessing\_matlab\_rbc.m} created in a previous exercise.

\item
Put the values for the parameters given in Table~\ref{tbl:RBC.Parameters} into a vector \texttt{MODEL.params}.

\item
Create two function handles:
\begin{lstlisting}[style=Matlab-editor,basicstyle=\mlttfamily\scriptsize]
exo_vars = 0;
obj     = @(xparam)     rbc_static_resid(xparam,exo_vars,MODEL.params)    ;
obj_ssq = @(xparam) sum(rbc_static_resid(xparam,exo_vars,MODEL.params).^2);
\end{lstlisting} 

\item
Provide initial guess values for the steady-state of all endogenous variables in a vector called \texttt{xparam0}.
Also create two vectors \texttt{LB} and \texttt{UB} with a lower and upper bound for the variables.

\item
Minimize either \texttt{obj} or \texttt{obj\_ssq} with different numerical optimization algorithms.
You might want to have a look into
\texttt{fsolve}, \texttt{lsqnonlin}, \texttt{fminunc}, \texttt{fminsearch}, \texttt{fmincon}, \texttt{simulannealbnd} and \texttt{patternsearch}.

\item
Compare the computed steady-state values, the residuals of the static model equations and the sum of squared residuals.

\item
Redo the exercise, but now randomize your starting values
  by drawing them uniformly given the lower and upper bounds.
That is, use the following code snippet:
\begin{lstlisting}[style=Matlab-editor,basicstyle=\mlttfamily\scriptsize]
randomize_initial_values = true; % set to true to randomize initial values
if randomize_initial_values
    no_good = 1;
    while no_good
        xparam0 = LB + (UB-LB).*rand(length(xparam0),1);
        if all(~isnan(obj(xparam0)))
            no_good = 0;
        end
    end
end
\end{lstlisting}
\end{enumerate}

\begin{solution}\textbf{Solution to \nameref{ex:RBCModelSteadyMATLAB}}
\ifDisplaySolutions%
\lstinputlisting[style=Matlab-editor,basicstyle=\mlttfamily\scriptsize,title=\lstname]{progs/matlab/steadystateRBC.m}
\fi
\newpage
\end{solution}