% !TEX root = endterm_exam_ss2022.tex
% !TeX encoding = UTF-8
% !TeX spellcheck = en_US
\documentclass{article}
%%%%%%%%%%%%%%%%%%%%%%%%%%%%%%%%%%%%%%%%%%%%%%%%%%%%%%%%%%%%%%%%%%%%%%%%%%%%%%%%%%%%%%%%%%%%%%%%%%%%%%%%%%%%%%%%%%%%%%%%%%%%%%%%%%%%%%%%%%%%%%%%%%%%%%%%%%%%%%%%%%%%%%%%%%%%%%%%%%%%%%%%%%%%%%%%%%%%%%%%%%%%%%%%%%%%%%%%%%%%%%%%%%%%%%%%%%%%%%%%%%%%%%%%%%%%
\usepackage[a4paper,top=2cm,bottom=2cm,left=2.5cm,right=2.5cm]{geometry}
\usepackage{amssymb,amsmath,amsfonts}
\usepackage[english]{babel}
\usepackage[a4paper]{geometry}
\usepackage{enumitem}
\usepackage{booktabs}
\usepackage{csquotes}
\usepackage{url}
\usepackage{graphicx}
\usepackage{booktabs,longtable}
\usepackage{xcolor}
%\parindent0mm
%\parskip1.5ex plus0.5ex minus0.5ex

\begin{document}
	
\title{Computational Macroeconomics}
\author{Endterm Exam}
\date{Summer 2022}
\maketitle


\begin{itemize}
\item
Answer \textbf{all} of the following \textbf{three} exercises in English.
\item
All assignments will be given the same weight in the final grade.
\item
Hand in your solutions before Sunday August, 10 2022 at 3pm.
\item
The solution files should contain your executable (and commented) Dynare and MATLAB files
  as well as all additional documentation as \texttt{pdf}, not \texttt{doc} or \texttt{docx}.
Your \texttt{pdf} files may also include scans or pictures of handwritten notes.
\item
Please e-mail ALL the solution files to \url{willi.mutschler@uni-tuebingen.de}.
I will confirm the receipt of your work also by email (typically within the hour). If not, please resend it to me.
\item
All students must work on their own, please also give your student ID number and the name of the module you want to earn credits for.
\item
It is advised to regularly check Ilias and your emails in case of urgent updates.
\item
If there are any questions, do not hesitate to contact Willi Mutschler.
\end{itemize}

%\section*{Changelog}
\newpage

\section[A Model of Consumption Choice with a Borrowing Constraint (Guerrieri and Iacoviello, 2015)]{A Model of Consumption Choice with a Borrowing Constraint (Guerrieri and Iacoviello, 2015)\label{ex:Guerrieri_Iacoviello_2015_borrcon}}
Occasionally binding borrowing constraints arise in a wide variety of models where households can \emph{self-insure} by holding and managing an asset,
  up to some borrowing limit, that can be used to buffer consumption against adverse shocks.
In these models, one can distinguish situations when a household is not constrained in the current period,
  and the traditional Euler equation for consumption holds;
  and situations when the household is credit constrained,
  current consumption is too low relative to next period,
  and the Euler equation for consumption does not hold.

Consider a consumer that maximizes:
\begin{align*}
E_t \sum_{s=0}^\infty \beta^s \frac{C_t^{1-\gamma}-1}{1-\gamma}
\end{align*}
subject to the budget constraint, and to an (occasionally binding) constraint stating that borrowing \(B_t\) cannot exceed a fraction \(m\) of income \(Y_t\):
\begin{align}
C_t + R B_{t-1} = Y_t + B_t \label{eq:borrcon:budget}
\end{align}
\begin{align}
B_t \leq m Y_t \label{eq:borrcon:constraint}
\end{align}
Above, \(R\) denotes the gross interest rate.
The discount factor \(\beta \) is assumed to satisfy the restriction that \(\beta R <1\),
  so that in the deterministic steady-state the borrowing constraint is binding.
Giving initial conditions, the impatient household prefers a consumption path that is falling over time,
  and attains this path by borrowing today.
If income is constant, the household will eventually be borrowing constrained and will roll its debt over forever,
  and consumption will settle at a level given by income less the steady-state debt service.
The log of income follows an AR{(1)} stochastic process of the form:
\begin{align}
\ln Y_t = \rho \ln Y_{t-1} + \epsilon_t \label{eq:borrcon:LOM_Y}
\end{align}
where \(\epsilon_t\) is an exogenous normally distributed innovation with standard deviation \(\sigma \).

\noindent~\\The goal of the following exercises is to simulate the model using Dynare's occbin toolkit.

\begin{enumerate}
\item
Denote with \(\lambda_t\) the Lagrange multiplier on the borrowing constraint~\eqref{eq:borrcon:constraint} and
  derive the following consumption Euler equation and Kuhn-Tucker condition:
\begin{align}
C_t^{-\gamma} = \beta R E_t C_{t+1}^{-\gamma} + \lambda_t \label{eq:borrcon:Euler}
\end{align}
\begin{align}
\lambda_t (B_t - mY_t) = 0 \label{eq:borrcon:KuhnTucker}
\end{align}
\end{enumerate}

The set of equations describing the system of necessary conditions for an equilibrium are given by a
  a system of four equations in four unknowns \({\{ C_t,B_t,\lambda_t,Y_t \}}\),
  which includes equations~\eqref{eq:borrcon:budget},~\eqref{eq:borrcon:LOM_Y},~\eqref{eq:borrcon:Euler},
  and the occasionally binding constraint~\eqref{eq:borrcon:KuhnTucker}.
Consider a case, when in the reference regime, the borrowing constraint binds,
  and the multiplier is greater than zero, i.e. \(B_t=mY_t\) and \(\lambda_t>0\).
In the alternative regime, the borrowing constraint is slack, and the multiplier is zero, i.e.\
  \(\lambda_t=0\) and \(B_t\leq m Y_t\).

\begin{enumerate}[resume]
\item
Write a Dynare mod file for this model, using the following calibration:
  \(\beta=0.945\), \(R=1.05\), \(\rho=0.9\), \(\gamma=1\), \(m=1\), and \(\sigma=0.0131\).
Compute the steady-state of all model variables.

\item
Do the following stochastic simulation (both linear and piecewise linear) using Dynare's occbin toolkit:
\begin{itemize}
\item Start from the non-stochastic steady-state where income is one.
\item In period 2, a two standard deviation shock brings up income unexpectedly.
\item In period 21, a two standard deviation shock pushes down income unexpectedly.
\end{itemize}

\item
Plot the reactions for \(t=1,\ldots ,40\) of
\begin{itemize}
\item consumption in percentage deviation from its steady-state
\item borrowing in percentage deviation from its steady-state
\item income in percentage deviation from its steady-state
\item the ratio of borrowing to income  
\end{itemize}
How long does the occasionally binding constraint bind in your simulation?
\end{enumerate}

\newpage

\subsection*{Codes and hints}

\begin{itemize}
\item The paper and appendix are available on Ilias.
\end{itemize}

\newpage

\section[Technology shocks with non-symmetric innovations]{Technology shocks with non-symmetric innovations\label{ex:Andreasen_2012}}
The goal of this exercise is to look into the third-order perturbation approach and replicate columns (2) and (3) of Table 2 in Andreasen (2012):
  \emph{On the effects of rare disasters and uncertainty shocks for risk premia in non-linear DSGE models},
  published in Review of Economic Dynamics (15).
For the following exercise, we will focus on the sections about rare disasters only.

\begin{enumerate}
\item
What is the main research question of the paper?
Why is the proposed third-order perturbation solution method appropriate to tackle this research question?

\item
Compared to the standard New Keynesian DSGE model we learned about in the lecture,
  what features are added in the paper and why?

\item
The third-order perturbation solution in Dynare's model framework is given by:
\begin{align*}
y_t &= \bar{y}\\
&
\colorbox{gray!10}{\(%
  +g_x (x_{t-1} - \bar{x}) + g_u u_{t}
\)}
\\
&
\qquad
\colorbox{gray!10}{\(%
  + g_{\sigma}
\)}
\\
&
\colorbox{gray!25}{\(%
  + \frac{1}{2} g_{xx} \left((x_{t-1} - \bar{x}) \otimes (x_{t-1} - \bar{x})\right)
  +             g_{xu} \left((x_{t-1} - \bar{x}) \otimes u_t\right)
  + \frac{1}{2} g_{uu} \left(u_t \otimes u_t\right)
\)}
\\
&
\qquad
\colorbox{gray!25}{\(%
  + g_{x\sigma} (x_{t-1} - \bar{x})
  + g_{u\sigma} u_t
\)}
\\
&
\qquad\qquad
\colorbox{gray!25}{\(%
  + \frac{1}{2} g_{\sigma\sigma}
\)}
\\
&
\colorbox{gray!50}{\(%
  + \frac{1}{6} g_{xxx} \left((x_{t-1} - \bar{x}) \otimes (x_{t-1} - \bar{x}) \otimes (x_{t-1} - \bar{x})\right)
  + \frac{1}{6} g_{uuu} \left(u_t \otimes u_t \otimes u_t\right)
\)}
\\
&
\qquad
\colorbox{gray!50}{\(%
  + \frac{3}{6} g_{xxu} \left((x_{t-1} - \bar{x}) \otimes (x_{t-1} - \bar{x}) \otimes u_t \right)
  + \frac{3}{6} g_{xuu} \left((x_{t-1} - \bar{x}) \otimes u_t \otimes u_t \right)
\)}
\\
&
\qquad\qquad
\colorbox{gray!50}{\(%
  + \frac{3}{6} g_{xx\sigma} \left((x_{t-1} - \bar{x}) \otimes (x_{t-1} - \bar{x}) \right)
  +             g_{xu\sigma} \left((x_{t-1} - \bar{x}) \otimes u_t \right)
  + \frac{3}{6} g_{uu\sigma} \left(u_t \otimes u_t \right)
\)}
\\
&
\qquad\qquad\qquad
\colorbox{gray!50}{\(%
  + \frac{3}{6} g_{x\sigma\sigma} (x_{t-1} - \bar{x})
  + \frac{3}{6} g_{u\sigma\sigma} u_t
\)}
\\
&
\qquad\qquad\qquad\qquad
\colorbox{gray!50}{\(
+ \frac{1}{6} g_{\sigma\sigma\sigma}
\)}
\end{align*}
where after running \texttt{stoch\_simul{(order=3)}}

\begin{itemize}
\item
\(x_t\) are the state variables and \(y_t\) are the variables that we focus our analysis on (i.e.\ reported or observed variables)

\item
the steady-states \(\bar{x}\) and \(\bar{y}\) can be accessed from the rows of\\
\texttt{oo\_.dr.ys{(oo\_.dr.order\_var)}}

\item
the \colorbox{gray!10}{first-order terms} can be accessed from the rows of\\
\texttt{oo\_.dr.ghx} and \texttt{oo\_.dr.ghu}

\item
the \colorbox{gray!25}{second-order terms} can be accessed from the rows of\\
\texttt{oo\_.dr.ghxx}, \texttt{oo\_.dr.ghxu}, \texttt{oo\_.dr.ghuu}, and \texttt{oo\_.dr.ghs2}

\item
the \colorbox{gray!25}{third-order terms} can be accessed from  the rows of \texttt{oo\_.dr.ghxxx},\\
\texttt{oo\_.dr.ghxxu}, \texttt{oo\_.dr.ghxuu}, \texttt{oo\_.dr.ghxss}, \texttt{oo\_.dr.ghuuu} and \texttt{oo\_.dr.ghuss}

\end{itemize}
Note that there is no \texttt{oo\_.dr.ghs}, \texttt{oo\_.dr.ghxxs}, \texttt{oo\_.dr.ghxus}, \texttt{oo\_.dr.ghuus}, and \texttt{oo\_.dr.ghs3}.

Why does Dynare not compute \(g_{\sigma}\), \(g_{xx\sigma}\), \(g_{xu\sigma}\), \(g_{uu\sigma}\), and \(g_{\sigma\sigma\sigma}\)?
\end{enumerate}

Now consider Table (2) of the paper.

\begin{enumerate}[resume]
\item
Replicate column (2) labeled \emph{Benchmark}. Note that this requires to run a simulation of length 2000000 for a third-order approximation to the model with Gaussian shocks.
In the hints below you will find out how to get the same shock series to replicate the numbers in the column.

\item
Replicate column (3) labeled \emph{Case I}. Note that this requires to run a simulation of length 2000000 for a third-order approximation to the model with non-symmetric shocks.
In the hints below you will find out how to get the same shock series to replicate the numbers in the column.

\item
Comment on the different results obtained in columns (2) and (3) of the table.
\end{enumerate}

\newpage

\subsection*{Codes and hints}

\begin{itemize}
\item
The paper and technical appendix are available on Ilias.

\item
The mod file \texttt{Andreasen\_2012\_rare\_disasters.mod} contains a Dynare version of the DSGE model used to study rare disasters.
This mod file simulates data and computes empirical moments from a simulated time series of length 2000000 for a first-order approximation to the model.
Note that your task is to adapt the codes that do the simulation for a third-order approximation.
The mod file also contains some further hints and explanations.

\item
The function \texttt{Andreasen\_2012\_get\_shocks.m} produces both the Gaussian and non-symmetric shock series used to replicate columns (2) and (3) in Table 2 of the paper.
It also creates \(\Sigma^{(3)}\) for both cases which is an input argument of \texttt{get\_ghs3.m}.

\item
The function \texttt{get\_ghs3.m} computes \(g_{\sigma\sigma\sigma}\) from
\begin{align*}
(A+B) g_{\sigma\sigma\sigma} = -\left( f_{y^{**}_{+}} g^{**}_{uuu} + f_{y^{**}_{+}y^{**}_{+}} (g^{**}_u \otimes g^{**}_{uu})P^3_{u\_uu} + f_{y^{**}_{+}y^{**}_{+}y^{**}_{+}} (g^{**}_u \otimes g^{**}_u \otimes g^{**}_u)\right)\Sigma^{(3)} \label{eq:gsss}
\end{align*}
where \(f_{y^{**}_{+}}\), \(f_{y^{**}_{+}y^{**}_{+}}\), and \(f_{y^{**}_{+}y^{**}_{+}y^{**}_{+}}\) are the first, second, and third-order derivatives of the dynamic model equations with respect to forward jumper variables \(y_{t+1}^{**}\).
\(g^{**}_u\), \(g^{**}_{uu}\), and \(g^{**}_{uuu}\) are the partial derivatives of the jumper variables with respect to shocks.
\(P^3_{u\_uu}\) is a permutation matrix.
\(\Sigma^{(3)} = E[u_t \otimes u_t \otimes u_t]\) are the vectorized unconditional third-order product moments.
\(A\) and \(B\) are the auxiliary perturbation matrices:
\begin{align*}
A & = f_{y_0} + \begin{pmatrix} \underbrace{0}_{n\times n_{static}} &\vdots& \underbrace{f_{y^{**}_{+}} \cdot g^{**}_{x}}_{n \times n_{spred}} &\vdots& \underbrace{0}_{n\times n_{frwd}}  \end{pmatrix}\\
B & = \begin{pmatrix} \underbrace{0}_{n \times n_{static}}&\vdots & \underbrace{0}_{n \times n_{pred}} & \vdots & \underbrace{f_{y^{**}_{+}}}_{n \times n_{sfwrd}} \end{pmatrix}
\end{align*}
where \(f_{y_{0}}\) are the first derivatives of the dynamic model equations with respect to all current endogenous variables \(y_{t}^{}\)
and \(g^{**}_{x}\) are the first partial derivatives of the jumper variables with respect to previous state variables.

\item
In MATLAB, the Kronecker product \(\mathcal{A} \otimes \mathcal{B} \otimes \mathcal{C}\) of arbitrary matrices \(\mathcal{A}\), \(\mathcal{B}\), and \(\mathcal{C}\)
  can be computed using \texttt{kron{(\(\mathcal{A}\),kron{(\(\mathcal{B}\),\(\mathcal{C}\))})}}.

\item
If your computer struggles with the simulation, reduce the length of the simulation to 200000 or lower.
On my MacBook Air (M1) it takes about a minute to do the simulation at third-order.
\end{itemize}

\newpage

\section[Solving the RBC model with leisure using a simple projection method]{Solving the RBC model with leisure using a simple projection method\label{ex:RBCLeisureProjection}}
Consider the basic RBC model with leisure and a CES utility function represented by the following system of equations:
\begin{align*}
{{C}}_{t}^{-{{\eta^C}}} &={{\beta}}\, E_t \left \{{{C}}_{t+1}^{-{{\eta^C}}}\, \left(1-{{\delta}}+{{R}}_{t+1}\right) \right \}
\\
{{W}}_{t}&=\frac{\psi \, {\left(1-{{L}}_{t}\right)}^{-{{\eta^L}}}}{{{C}}_{t}^{-{{\eta^C}}}}
\\
{{W}}_{t}&=\left(1-{{\alpha}}\right)\, {{A}}_{t}\, {\left(\frac{{{K}}_{t-1}}{{{L}}_{t}}\right)}^{{{\alpha}}}
\\
{{R}}_{t}&={{\alpha}}\, {{A}}_{t}\, {\left(\frac{{{K}}_{t-1}}{{{L}}_{t}}\right)}^{{{\alpha}}-1}
\\
{{Y}}_{t}&={{A}}_{t}\, {{K}}_{t-1}^{{{\alpha}}}\, {{L}}_{t}^{1-{{\alpha}}}
\\
{{Y}}_{t}&={{C}}_{t}+{{I}}_{t}
\\
{{K}}_{t}&={{I}}_{t}+\left(1-{{\delta}}\right)\, {{K}}_{t-1}
\\
\log\left({{A}}_{t}\right)&={{\rho}}\, \log\left({{A}}_{t-1}\right)+{{\varepsilon}}_{t}
\end{align*}
The description of variables and parameters are given in Tables~\ref{tbl:RBC.Variables} and~\ref{tbl:RBC.Parameters}.
The goal of this exercise is to solve the model with a simple projection method.

\begin{enumerate}
\item
Briefly describe the four steps of the projection method to solve a DSGE model.

\item
What is the \emph{curse of dimensionality}? Does this curse also apply to perturbation methods?

\item
Solve the model using a projection method. To this end:

\begin{itemize}
  \item
  Set up a simple grid for the state variables \(K_{t-1}\) and \(A_t\):
  \begin{itemize}
    \item For \(K_{t-1}\) use 10 evenly distributed points between \(\frac{1}{2}\bar{K}\) and \(\frac{3}{2}\bar{K}\), where \(\bar{K}\) is the steady-state of \(K_t\).
    \item For \(\log(A_t)\) use 10 evenly distributed points between \(-3\frac{\sigma^2}{1-\rho^2}\) and \(3\frac{\sigma^2}{1-\rho^2}\).
    \end{itemize}

  \item
  Use regular second-order polynomials for approximating the policy functions for the log of consumption and the log of labor supply:
  {\small
  \begin{align*}
  \log(C_t) &= \theta_{1} + \theta_{2} \log{(K_{t-1})} + \theta_{3} \log{(A_t)} + \theta_{4} \log{(K_{t-1})}^2 + \theta_{5} \log{(A_t)}^2 + \theta_{6} \log{(K_{t-1})}\log{(A_t)}
  \\
  \log(L_t) &= \theta_{7} + \theta_{8} \log{(K_{t-1})} + \theta_{9} \log{(A_t)} + \theta_{10} \log{(K_{t-1})}^2 + \theta_{11} \log{(A_t)}^2 + \theta_{12} \log{(K_{t-1})}\log{(A_t)}
  \end{align*}}

  \item
  To minimize the Euler residuals use {\texttt{fminsearch}} with the following set of options:\\
  {\small
  \texttt{optimset{({'MaxFunEvals'},{1e5},{'MaxIter'},{1e6},{'TolFun'},{1e-7},{'TolX'},{1e-7},{'Display'},{'Iter'})}}
  }.
\end{itemize}

\item
Plot both the consumption and labor supply choice as a function of the grid on \(K_{t-1}\) and for 3 values of \(A_t = \{0.9,1.0,1.1\} \).
\end{enumerate}

\newpage

\subsection*{Codes and hints}

\begin{itemize}
\item
A Dynare version of this model is given in the file \texttt{rbc\_ces.mod}.
This file computes the steady-state and a second-order perturbation approximation.
It might contain some useful information or computations for the projection method.
\end{itemize}

\begin{center}
\begin{longtable}{cc}
\caption{Variables}\label{tbl:RBC.Variables}\\%
\toprule%
\multicolumn{1}{c}{\textbf{Variables}} &
\multicolumn{1}{c}{\textbf{Description}}\\%
\midrule\midrule%
\endfirsthead%
\multicolumn{2}{c}{{\tablename} \thetable{} {--} Continued}\\%
\midrule%
\multicolumn{1}{c}{\textbf{Variables}} &
\multicolumn{1}{c}{\textbf{Description}}\\%
\midrule\midrule%
\endhead%
\({Y_t}\) & output\\
\({C_t}\) & consumption\\
\({K_t}\) & capital\\
\({L_t}\) & labor\\
\({A_t}\) & productivity\\
\({R_t}\) & real return on capital\\
\({W_t}\) & real wage\\
\({I_t}\) & investment\\
\({\varepsilon_t}\) & Productivity Shock\\
\bottomrule%
\end{longtable}
\end{center}

\begin{center}
\begin{longtable}{ccc}
\caption{Parameters}\label{tbl:RBC.Parameters}\\%
\toprule%
\multicolumn{1}{c}{\textbf{Parameter}} &
\multicolumn{1}{c}{\textbf{Value}} &
\multicolumn{1}{c}{\textbf{Description}}\\%
\midrule%
\endfirsthead%
\multicolumn{3}{c}{{\tablename} \thetable{} {--} Continued}\\%
\midrule%
\multicolumn{1}{c}{\textbf{Parameter}} &
\multicolumn{1}{c}{\textbf{Value}} &
\multicolumn{1}{c}{\textbf{Description}}\\%
\midrule%
\endhead%
\({\alpha}\) 	 & 	 0.350 	 & 	 Output Elasticity of Capital\\
\({\beta}\) 	 & 	 0.990 	 & 	 Discount Factor\\
\({\delta}\) 	 & 	 0.025 	 & 	 Depreciation Rate\\
\({\rho}\) 	   & 	 0.900 	 & 	 Persistence technology\\
\({\bar{L}}\)  & 	 0.333 	 & 	 steady-state labor that calibrates \(\psi \) endogenously\\
\({\eta^C}\) 	 & 	 2.000 	 & 	 CES utility elasticity with respect to consumption\\
\({\eta^L}\) 	 & 	 1.500 	 & 	 CES utility elasticity with respect to labor\\
\({\psi}\) 	   & 	 \(\bar{W}{(1-\bar{L})}^{\eta_L} \bar{C}^{-\eta_C}\) 	 & 	the labor disutility weight, where \(\bar{W}\) and \(\bar{C}\) \\
               &           &are the steady-states of \(W_t\) and \(C_t\)
\\
\bottomrule%
\end{longtable}
\end{center}

  

\end{document}
